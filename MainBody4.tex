\section{MainBody4}

In the former chapter MACs and digital signatures were discussed one way to 
JSON
"JSON Web Signature (JWS) represents content secured with digital signatures or Message Authentication Codes (MACs) using JSON-based  data structures" [rfc7515].
One aufgabenbereich/Möglichkeit  von JWS ist das gewährleisten der Integrität von nachrichten by using Macs or signatures im MQTT Protokoll.(One possebility to use JWS is for providing integrity of messages(namnen googlen) in MQTT).
A valid JWS consists of these following components : The JOSE Header, the JWS Payload and the JWS Signature.\newline

The \textbf{JOSE Header} (JSON Object Signing and Encryption ) is a JSON object consisting of a set of Header parameter, which defines the parameters and the cryptographic operations to use to secure the JWS Payload. There are two types of headers, which combined form the JOSE Header:\newline The \textbf{JWS Protectet Header} and the \textbf{JWS Unprotectet Header}.\newline
Both are JSON objects, but the difference is, the JWS Protected Header contains the integrity protected Header Parameters whereas the JWS Unprotectet Header contains the not integrity protectet parameters *besser unprotected?*.\newline
An example for a header parameter is the "alg" (algorythm) Header Parameter which must be present in the JOSE Header."(This parameter) identifies the cryptographic algorithm used to secure the JWS. the JWS Signature is not valid if the "alg" value value does not represent a supported algorithm or if there is not a key for use with that algorithm associated with the party that digitally signed or MACed the content" [rfc7515].\newline

The \textbf{JWS Payload} can be describet as "(t)he sequence of octets to be secured -- a.k.a. the message. It can be any arbitrary  sequence of octets"[rfc7515].\newline
The claims in a JWT, encoded as a JSON object, are often used as payload. [rfc7519]\newline
Two examples for claim names, both not mandatory, but very usefull, are "iss" and "exp". "The iss (issuer) identifies the principal that issued the JWT" [rfc7519] and "the "exp" (expiration time) identifies when the JWT MUST NOT be accepted for processing "[rfc7519] \newline

The \textbf{JWS Signature} "is the computation of a Mac or digital signature over the JWS Protected Header and the JWS Payload".[rfc7515]\newline 

The JWS can be further characterized by its two serialisations: The \textbf{JWS Compact Serialization} and the \textbf{JWS JSON Serialization}. 
In the JWS Compact Serialization the JWS is represented as a compact, URL.safe string and can be portrayed in the following form :\newline    
BASE64URL(UTF8(JWS Protected Header)).BASE64URL(JWS Payload).BASE64URL(JWS Signature) [mediumfalcon...]
In this serialization, the JWS Protected Header is used as the JOSE Header. Therefore, no JWS Unprotected Header will be used.

On the contrary to the JWS JSON Serialization , the JWS Compact Serialisation provides no syntax for an JWS Unprotectet Header and only supports one digital signature/MAC.
In the JWS JSON Serialization there are two related syntaxes: a flattered syntax,to secure content with only one MAC/digital signature and a fully general syntax, for securing with more than one MAC/digital signature.[rfc7515] In this section only the fully general syntax is threatend. 
A JWS in the fully general JWS JSON Serialization" includes 2 top-level elements: payload and signatures (which is a JSON array), and three sub elements under each entry of the signatures array: protected, header and signature" [falcon..]
In each entry at least either the value "protected" or the value "header" MUST be present to guarantee the "alg" Header Parameter is present.[rfc7515]\newline
An example for this concept can be displayed as follows:\newline 
	" $\{$ "payload":"$<$payload contents $>$",\newline    "signatures":[\newline
   $\{$"protected":"$<$integrity-protected header 1 contents$>$",\newline
   "header":$<$non-integrity-protected header 1 contents$>$, \newline
   "signature":"$<$signature 1 contents$>$"$\}$,\newline ...\newline
   $\{$"protected":"$<$integrity-protected header N contents$>$",\newline
   "header":$<$non-integrity-protected header N contents$>$,\newline"signature":"$<$signature N contents$>$"$\}$] \newline    $\}$" [rfc7515]

For both types of serializations the computation of a JWS is similar. To create a valid JWS the following steps have to be performed: 
\begin{enumerate}
\item "Create the content used as payload" [rfc7515]
\item Compute the Base64Url-encodet payload 
\item "Create the JOSE Header with the desired set of Header Parameter as a JSON object." [rfc7515]
\item Compute the utf8 representation of the JWS Protected Header and base64url-encode it.
\item Create the signature by using the algorithm, defined in "alg", over  ASCII(BASE64URL(UTF8(JWS Protected Header)).BASE64URL(JWS Payload), which is defined as the JWS Signing Input. If "alg" has the value "none", there will be no signature.
\item Compute the encoded signature: BASE64URL(JWS Signature)
\item For the JWS JSON Serialization the steps 3-6 have to be repeated for each MAC/digital signature defined in the JWS JOSE Header.
\item Compute the specific serialization output. For the structure look at the description of JWS Compact Serialization and JWS JSON Serialization.
\end{enumerate} [rfc7515]



Like the computation of a JWS, the validation of a JWS is again similar for both types of serialization.Its important to note that if there are multiple signatures, like in the fully general JWS JSON Serialization,  its application-specific how many and which JWS Signature must be validated successfully.[rfc7515] However, at least one JWS Signature MUST be validated successfully.
To validate a JWS following steps have to be performed:\newline  
\begin{enumerate}
\item "Parse the JWS representation to extract the serialized values for the components of the JWS."[rfc7515] Look at the specific serialisation for the specific components.
\item Decode the base64url-encoded JWS Protected Header.
\item Verify that the result of step 2 is a uft8-encoded, valid JSON object.
\item For the JWS Compact Serialisation the JWS Protected header is used as the JOSE Header. For the JWS JSON Serialisation the JWS Protected Header Members and the JWS Unprotected Header Members combinded build the JOSE Header .
\item Verify that all fields, which are required to be supported,are understood and can be processed by the implementation.

\item Decode the base64url-encoded JWS Payload.
\item Decode the base64url-encoded JWS Signature.
\item Validate if the JWS Singning input maced/digitally signed conformed the JWS Signature.
\item For the JWS JSON Serialisation the steps 4-8 have to be repeated for each MAC/digital signature defined in the JWS JOSE Header.
\item The JWS MUST be considered invalid if all validations in step 9 failed. (In the case of JWS Compact Serialization it is easy to indicate whether the JWS was validated succesfully or not, by the result, whereas in the case of the JWS JSON serialisation the result only indicates which valididation was successfull and which not.)
\end{enumerate}
------------------------

Schluss 
How precise a JWS can provide integrity depends on which and how much MACs/digital signatures are beeing used. If there is no algorithm, integrity cannot be guaranteed. In MQTT 
b the usage of the JWS it is important to remember, that the payload is only BASE64URL encoded and therefore the it is easy to read the message.