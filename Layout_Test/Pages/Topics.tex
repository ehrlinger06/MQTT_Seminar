\section{Avalible Topics}
In this section, we can collect the topics given to us by Dr. Poehls. Every Teammember has to choose one of the provided topics to work one and fill a few pages with it.

\subsection{Intro}
When information is stored or transmitted using an unreliable or unsafe medium, it is important to be able to check the integrity of the information. In an MQTT-environment it is not uncommon to transmit data through an unreliable network, so using any kind of method to provide authentication for the sent data is recommended. In the MQTT-protocol itself is, by default, no message authentication mechanism included, that means when you need to be able to authenticate a message the needed mechanise has to build in manually afterwards. The commonly known way for being able to authenticate, if a message was received containing the right information, is to compute a piece of additional information and sending it with the message. Both sender and receiver need to be able to compute this information, so they need to share the secret, how to compute this information. This secret contains a base formula as well all additional information needed. The additional piece of information is computed with using the actually sent message as an input, so the information cannot be recreated, when the information got changed during transmitting. This principal of using additional information next to the actual message to give a possibility of evaluating a received message for integrity, is called a message authentication code, short MAC.The MAC is first computed by the sender of the message and then sent to the receiver with the actual message, the receiver uses the received message to compute the MAC by himself, only if the two calculated MACs match the receiver will know, that  he received the actual message, the sender wanted him to receive. Because there is more than way to compute such a MAC, in the following we will evaluate four of this way in more detail. Now we will first present the scientific work, which has already been done with this topic, afterward we will present, as previously mentioned, four different methods for generating a MAC. 
\newline
(Here I would write all topics in the order they actually have in the final paper)
\newline
a
\newline
b
\newline
c
\newline
In the end, we will conclude, which of the presented methods is the most suitable method for generating MACs in an MQTT- environment.
