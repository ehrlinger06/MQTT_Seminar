\section{Related Work}

\subsection{Message Authentication Codes based on Hash-Functions using HMAC's}

The basic idea of HMAC’s gets defined by two standards, one provided by Request for Comments over the Internet Society (ISOC) \cite{RFC} and the other on provided by the American department of commerce as a federal information processing standards publication in the category computer security: cryptography \cite{FIBS}. Another notable standard is also provided by Request for Comments over the Internet Society (ISOC), it lists several examples for input into an HMAC generating function and the corresponding output \cite{RFC2}. The necessity to include an integrity mechanism like HMAC is also stated in the paper "Vulnerabilities and Limitations of MQTT Protocol
Used between IoT Devices" published by "Dan Dinculeană and Xiaochun Cheng" \cite{LIMI}.