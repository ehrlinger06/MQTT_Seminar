\section{Indroduction}

When information is stored or transmitted using an unreliable or unsafe medium, it is essential to be able to check the integrity of the information. In an MQTT-environment, it is not uncommon to transmit data through an unreliable network, as the design considered such networks. The Message Queuing Telemetry Transport (MQTT) protocol is designed as a machine-to-machine (M2M) protocol. The protocol is like the HTTP-protocol, a Client-Server-protocol, which is used to connect Clients over a special server called a broker. The MQTT protocol distributes information using a publish/subscribe-pattern \cite{LIMI}. This pattern helps cutting down the required header-information, as a subscriber sends the information to the broker, without knowing who receives the sent information. The receiver of the sent information is only known by the broker, which distributes the information to every subscriber of the topic the information belongs to. Using this pattern for a protocol brings several advantages. One advantage would be the separation of sender and receiver, so the receiver or receivers can change without interfering with the publisher, another advantage is, that the use of a broker makes it unnecessary for the publisher to include complex receiver-information with the sent packet. The missing need for receiver-information also helps by cutting down on the amount of data in the packet header. \\
In fact, the MQTT protocol is so lightweight, the header is only 2 bytes big \cite{IBM}. The downside of such a small header is that it lacks some security enhancing features, like for example some kind of integrity feature. In an IT-security aspect the word integrity can have several definitions, like “Integrity means the property that data or information have not been altered or destroyed in an unauthorized manner”\cite{INTI1} or " ”Data Integrity" means the confirmation that data which has been sent, received, or stored are complete and unchanged.” \cite{INTI2}. For achieving integrity, we look at three different algorithms and one framework in this paper. After the related work in section two, in section three we take a look at Message authentication codes using hashing, more precise HMAC, in section four we also take a look Message authentication codes using hashing, but this time using Merkle trees. In section five, we examine the use of the elliptic-curve Digital Signature Algorithm to create digital signatures and in section six, we look at JSON Web Signature (JWS)-framework. After that, in section seven, we evaluate the methods, we looked at in this paper, and present our recommendation.
