\section{Message Authentication Codes based on Hash-Functions using HMAC's}

In this section, we will look at and evaluate the use of Hash-functions on the Message authentication code sent with messages in an MQTT environment. \\
%Message authentication code%
A message authentication code (MAC) gets send with the message from the sender to the receiver so that the receiver can authenticate the content of the received message. \\
%Why use mac with hash in the first hand?%
Hash-Functions find wide use in constructing a MAC. The reasons for this are a faster execution time than for symmetric block chippers, an essential feature in low power IoT-devices. A second reason is good availability for library code providing suitable hashing algorithms, so used functions are known to work as intended. 
%Def hash-function%
\subsubsection{Definition of HMAC}
Before going more into detail, I want to clarify what is meant by a hash function. ''We assume H to be a cryptographic hash function where data is hashed by iterating a basic compression function on blocks of data. We denote by B the byte-length of such blocks, and by L the byte-length of hash outputs. The authentication key K can be of any length up to B, the block length of the hash function. Applications that use keys longer than B bytes will first hash the key using H and then use the resultant L byte string as the actual key to HMAC. In any case, the minimal recommended length for K is L bytes. [\dots] We define two fixed and different strings ipad and opad as follows (the ’i’ and ’o’ are mnemonics for inner and outer)
\begin{center}
ipad = the byte 0x36 repeated B times \\
opad = the byte 0x5C repeated B times.
\end{center}
To compute HMAC over the data ‘text’ we perform 
\begin{center}
H(K XOR opad, H(K XOR ipad, text))''
\end{center}
\cite{RFC}.  \\
% This definition can be used only for hash-functions, which require a secret key to compute a hash value. The use of a key is necessary, because without a secret key, everyone, who knows the hash function, used to compute the MAC, could compute a valid MAC according to the actually invalid message and therefore making the use of a MAC irrelevant. \\

%Using underlying hash functions%
When using a hash function to compute a MAC, the underlying hash-function is used as a black-box \cite{KHF}, so that the function itself could be swapped out without affecting the rest of the code. The reasons for keeping the use of the actual hash-function so flexible are, ease of maintainability, ease of testing and ease of upgradeability. The first two are connected because its recommended to choose a hash function, where the code is open-source, which gives the piece of code support, so possible threats are detected and bugs get fixed sooner. The ease of upgradeability comes through the encapsulation of the code, done to achieve the black-box effect, which supports the already mentioned swapping of hash-functions.
The most commonly used hash-functions to generate a MAC are either MD5 or SHA-1 (\cite{CRY-MD5}, \cite{NPC}), for all of these are implementations available and they are widely adopted. Of course, when using any kind of function to achieve more security, it is essential to know about possible flaws of the used function. 
%Security consideration of underlying hash functions%
\subsubsection{Security of underlying hash function}
Both MD5 and SHA-1 are cryptographic hash function, so their basic mechanism is to compress an input I to an output O of a specific length L. L is only influenced by the function itself, not by the input. In case of MD5, this output O is 128 bits or 16 bytes \cite{BAV} long when using SHA-1 the output is 160 bits or 20 Bytes \cite{BAV} long. Any hash-function used in a security-sensitive area should be considered secure, that a hash-function is considered secure, it needs to have preimage resistance, second-preimage resistance and collision resistance. The relation between these three resistances is the following, ''collision resistance implies second preimage resistance and second preimage resistance implies preimage resistance'' \cite{SPR-res}. Through this implication chain, it can be concluded, that when collision resistance gets refuted, this also falls back to second preimage resistance and preimage resistance. Unfortunately, both hash-functions, MD5(\cite{COL1}, \cite{COL2}) and SHA-1(\cite{COL3}, \cite{COL4}), are known to have collisions. At first glance, these found collisions seem to make MD5 as well as SHA-1 a bad choice for generating HAMCs, as HMAC can only be defined as secure, when the underlying hash-function was weakly collision resistant, ''The results are for NMAC but can be lifted to HMAC. It is shown in [3] that NMAC is a secure PRF if (1) the underlying compression function h is a secure PRF, and also (2) that the hash function H is weakly collision resistant (WCR).'' \cite{HMAC-SEC1}(The here mentioned NMAC stands for Nested MAC, which is an othther way to constructed MACs using hash-functions). This means, if there are collisions for the underlying hash-function, the resulting HMACs cannot be considered secure, but this definition changed, it was proven, ''that NMAC is a PRF under the sole assumption that the underlying compression function h is itself a PRF. In other words, the additional assumption that the hash function is WCR is dropped. (And, in particular, as long as h is a PRF, the conclusion is true even if H is not WCR, let alone CR.)'' \cite{HMAC-SEC1}. This last statement speaks only about NMAC, but as it was stated in a prior statement, ''The results are for NMAC but can be lifted to HMAC.'' \cite{HMAC-SEC1}. So even though both MD5 and SHA-1 can no longer be considered collision resistant, as shown above, this does not affect their use as an underlying hash function for computing HMAC's.
%Angriffsarten für hmac% 
\subsubsection{Attack vectors}
Even now, when collisions in the underlying hash-functions are no longer compromising the security of HMACs, there are other possible attacks, which could compromise the security and so also the use of HMAC. The attack vectors know to HMAC are called distinguishing attack and forgery attack. The distinguishing attacks divide in distinguishing-H and distinguishing-R attacks, the difference between the two is, ''Distinguishing-R attack means distinguishing a MAC from a random function, and distinguishing-H attack detects an instantiated MAC (by an underlying hash function or block cipher) from a MAC with a random function.''\cite{ATT1}. A distinguishing-R attack can be made using the so-called birthday paradox. It requires about 2\textsuperscript{(n$/$2)} messages (n is the length of the output of the hash function) and works with a probability of 63\%\cite{ATT1}. The most famous example of the birthday paradox is that in a room with a minimum of 23 people, there is a chance from up to 86\% that two people share the same birthday not considering their year of birth. This probability seems to be very high for the number of needed messages, but even after a successful distinguish-R attack there is a single random HMAC compromised, the to compromise HMAC cannot be chosen beforehand, so the actual probability of distinguishing a useful message can be considered low, the actual probability depends on the used underlying hash-function. The probability that a distinguish-H attack is successful is about 2\textsuperscript{-n}\cite{ATT1} (n is the length of the output of the hash function). This probability again depends on the length of the output of the underlying hash function. The probability of a successful distinguish-H attack needs to be so low as it is, because here a preselected message gets distinguished, so the use of a single successful attack is much higher. 
%Key length/Hashlength%
\subsubsection{Keylength, Hashlength}
When computing an HMAC value, the definition states, that there is a specific key used, which also serves as a shared secret between sender and receiver. This key should be chosen randomly. The suggestion is that a key should be as long as a block of the underlying hash function, it is stated that such a block is 64 byte for both MD5 and SHA1 \cite{RFC}. Longer keys do not increase the strength directly but can help when the random-function used for key generation is weak. Shorter keys are also possible, but the minimum size is the length of the output of the hash-function \cite{RFC}. There is no clear guideline for key replacement given, as there are no known practical attacks, which could exploit such a key \cite{RFC}. \\
As already stated, the length of the output depends on the hash function itself, MD5 produces a 128 bit or 16 bytes hash and SHA-1 produces a 160 bit or 20 bytes hash. \\
%Truncated output%
''A well-known practice with message authentication codes is to truncate the output of the MAC and output only part of the bits'' \cite{RFC}. The basic idea of truncation is to shorten the output produced by the hash function. This brings both advantages and disadvantages. An advantage would be given less information to a possible attacker. A disadvantage would be that there are lesser bits that need prediction when attacking. Cutting away more than n/2 (n is the length of the output of the hash function)is not recommended, because otherwise, it would match a boundary for the birthday-attack as mentioned above. 
%Computing time comparison(material)%
\subsubsection{Computing time}
HMAC’s are used as a security enhancing feature, but next to the security features speed plays an essential role because the best security is beginning to seem useless when securing a packet is by far the most complicated thing in the whole chain. The speed is given in Throughput, so how much data can be processed per time unit when using MD5 the throughput is 756,0 MB/S \cite{MAX}, this comes down to 12096 message blocks per second or 0,08ms (=milliseconds) per block. When using SHA-1, the throughput goes up to 1024,0 MB/s \cite{MAX}, this comes down to 16384 message blocks per second or 0,06ms.



